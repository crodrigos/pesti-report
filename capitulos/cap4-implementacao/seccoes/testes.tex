\section{Testes}

A prática de testes é fundamental para assegurar o correto funcionamento de qualquer programa, permitindo verificar se o código implementado cumpre as especificações definidas. Esta secção pretende apresentar e detalhar os diferentes tipos de testes realizados ao longo do projeto e as tecnologias utilizadas para este.

\subsection{Testes Unitários}

Os testes unitários constituem uma prática essencial no desenvolvimento de \textit{software}, cujo principal objetivo é garantir que cada unidade de código se comporta conforme esperado. Esta abordagem permite a deteção precoce de erros, aumenta a fiabilidade do \textit{software} e facilita a manutenção e evolução do sistema. Para a execução destes testes, deu-se prioridade à utilização do \textit{JUnit}, em conjunto com as ferramentas disponibilizadas pelo \gls{Spring}.

\begin{lstlisting}[language=Java, label={lst:class-CompanyTest}, caption={Class \textit{CompanyTest} - Exemplificação de testes Unitários}]
@DataJpaTest()
class CompanyTest {

    @Autowired
    private TestEntityManager entityManager;

    @Autowired
    private CompanyRepository companyRepository;

    @Test
    void shouldPersistAndLoadCompanyWithValueObjects() {
        Company company = new Company();
        company.setName(new CompanyName("OpenAI"));
        company.setDescription(new CompanyDescription("Artificial Intelligence Research"));

        entityManager.persist(company);
        entityManager.flush();
        entityManager.clear();

        Optional<Company> found = companyRepository.findById(company.getIid());

        assertThat(found.isEmpty()).isFalse();
        assertThat(found.get().equals(company));
    }
    |\Suppressnumber|
    ...
    |\Reactivatenumber|
}
\end{lstlisting}

\subsection{Testes de Implementação}

Para os testes de implementação, deu-se especial ênfase à utilização da ferramenta Postman, a qual proporciona um ambiente colaborativo e estruturado para a testagem de \gls{API}.

A figura \ref{lst:postman-post-company} apresenta o script pós-teste executado pelo postman para verificar a resposta ao pedido do tipo POST para "/company". 

\begin{lstlisting}[style=Javascript, label={lst:postman-post-company}, caption={Script de test da rota POST /company}]
pm.test("Status code is 201", function () {
    pm.response.to.have.status(201);
});

pm.test("Response has correct company data", function () {
    var jsonData = pm.response.json();
    pm.expect(jsonData.name.value).to.eql("OpenAI");
    pm.expect(jsonData.description.value).to.eql("Artificial Intelligence Research");
    pm.expect(jsonData.id).to.exist;
})

const companyId = pm.response.json().id;
await pm.sendRequest({
    url: `${pm.environment.get("localhost")}/company/${companyId}`,
    method: 'DELETE',
    header: {
        'Content-Type': 'application/json'
    }
});
\end{lstlisting}

