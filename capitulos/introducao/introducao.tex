\chapter{Introdução}
\label{chap:introducao}

\section{Enquadramento}
\label{sec:introducao_enquadramento}

O projeto desenvolvido com o curso Blended Mobility

O foco principal será o desenvolvimento de um projeto no âmbito do curso BlendED, que se caracteriza pela integração da aprendizagem em mobilidade num contexto de trabalho híbrido. Este curso promove a colaboração entre alunos de várias universidades internacionais permitindo a combinação de atividades presenciais e online, potencializando a flexibilidade e a personalização do processo de aprendizagem. Todas as equipas envolvidas terão de desenvolver, ao longo de quatro meses, um projeto para uma empresa parceira, aplicando metodologias ativas e colaborativas típicas do ensino híbrido.

Numa primeiro fase, todos os membros deslocaram-se se até ao Instituto Universitário da Maia (ISMAI), onde se deu uma semana para conhecer toda a equipa, contextualização do projeto e estruturação do trabalho para os seguintes meses. Durante os seguintes quatro meses ocorrereu o desenvolvimento do projeto. No final, todas a equipas irão reuiniram na Universidade de Trier, na Alemanha, para a apresentação final do que foi desenvolvido.



\section{Descrição do Problema}
\label{sec:introducao_descproblema}

O Website do curso Blended4Future estava muito àquem do espectado, muitos elementos não seguiam um design inconsistente e antiquado, ou não estavam completamente funcionais. 

A organização desejava uma plataforma onde se pudesse autmaticamente adicionar projetos, alunos, universidades e empresas em uma só plataforma. Por tal foi colocada um proposta de desenvolvimento de uma nova aplicação web que substituiria esta anterior. 



\section{Objetivos}

A aplicação web a desenvolver deverá incluir um sistema de autenticação com diferenciação entre perfis de utilizador, nomeadamente administradores e utilizadores comuns, assegurando uma gestão adequada de permissões. 

Adicionalmente, deverá permitir a criação de novos projetos e a associação de diferentes intervenientes a cada um, consoante o seu papel. Para além disso, deverá ser implementada uma biblioteca de projetos, acessível a qualquer utilizador da plataforma, onde estarão disponíveis todos os projetos desenvolvidos no âmbito do curso, promovendo a sua consulta e divulgação.

A interface da aplicação deverá seguir as diretrizes de design definidas previamente por um membro da equipa dedicado ao design visual da aplicação.



\section{Abordagem}

A equipa era constituida de um conjunto de alunos de varias universidades europeias e tinha como constituição:

\begin{itemize}
    \item 5 Desenvolvedores
    \item 1 Designer 
    \item 1 Marketer 
\end{itemize}

Para uma melhor estruturação foi escolhida a metodologia de trabalho Scrum com cada sprint com uma duração de 2 semanas. Foi ainda estabelicido como grupo a necessidade de reuniões semanais para atualização das tarefas a serem feitas e estimulação de um ambiente mais comunicativo.
