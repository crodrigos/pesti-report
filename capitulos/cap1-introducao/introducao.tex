\chapter{Introdução}
\label{chap:introducao}

\section{Enquadramento}
\label{sec:introducao_enquadramento}

Este relatório foi desenvolvido com base num projeto enquadrado no âmbito da unidade curricular de Projeto Estágio (PESTI) da Licenciatura em Engenharia Informática (LEI) do Instituto Superior de Engenharia do Porto (ISEP)

Este projeto ocorreu no enquadramento do Projeto BlendED (também referido como Blended4Future ou BlendedMobility\footnote{https://blendedmobility.com}). Este curso dá a estudantes a oportunidade de desenvolverem as suas \textit{soft} e \textit{hard skills} num projeto com alunos de diferentes culturas e países trabalhando num projeto para empresas reais

\subsection{Apresentação da organização}

BlendedMobility é uma iniciativa educativa internacional que promove o desenvolvimento de projetos colaborativos no contexto do ensino híbrido. Esta visa combinar experiências de aprendizagem presencial e online, proporcionando aos estudantes uma formação mais flexível, personalizada e orientada para a prática.

O programa reúne alunos de diferentes universidades europeias que, ao longo de quatro meses, trabalham em conjunto no desenvolvimento de projetos reais para empresas parceiras. A metodologia adotada assenta em práticas ativas e colaborativas, potenciando competências essenciais como trabalho em equipa, comunicação intercultural e resolução de problemas num ambiente profissional simulado.

O percurso inicia-se com uma semana presencial no Instituto Universitário da Maia (ISMAI), onde as equipas se conhecem, recebem o enquadramento do projeto e planeiam as etapas de trabalho. O desenvolvimento dos projetos decorre num regime híbrido, combinando sessões online e trabalho autónomo. Ao final do ciclo, todas as equipas reúnem-se presencialmente na Universidade de Trier, na Alemanha, para apresentar os resultados dos seus projetos a um painel de avaliadores e representantes das empresas envolvidas.


\section{Descrição do Problema}
\label{sec:introducao_descproblema}

O Website do curso Blended4Future estava muito aquém do espectado, muitos elementos seguiam um design inconsistente e antiquado, ou não estavam completamente funcionais. 

A organização desejava uma plataforma onde se pudesse autmaticamente adicionar projetos, alunos, universidades e empresas em uma só plataforma. Por tal foi colocada um proposta de desenvolvimento de uma nova aplicação web que substituiria esta anterior. 

Nesta aplicação, estudantes professores e representantes de empresas poderiam ver os projetos em que estavam envolvidos, fazer posts sobre os seus respetivos projetos.

Foi ainda requisitado uma maneira de ver todas a edições do Blended4Future e todos o projetos neste envolvido

\section{Objetivos}

A aplicação web a desenvolver deverá incluir um sistema de autenticação com diferenciação entre perfis de utilizador, nomeadamente administradores e utilizadores comuns, assegurando uma gestão adequada de permissões. 

Adicionalmente, deverá permitir a criação de novos projetos e a associação de diferentes intervenientes a cada um, consoante o seu papel. Para além disso, deverá ser implementada uma biblioteca de projetos, acessível a qualquer utilizador da plataforma, onde estarão disponíveis todos os projetos desenvolvidos no âmbito do curso, promovendo a sua consulta e divulgação.

A interface da aplicação deverá seguir as diretrizes de design definidas previamente por um membro da equipa dedicado ao design visual da aplicação.


\section{Abordagem}
\subsection{A equipa}

A equipa foi composta por um grupo de estudantes provenientes de várias universidades europeias, com a seguinte constituição:

\begin{itemize}
    \item \textbf{5 Desenvolvedores};
    \item \textbf{1 Designer};
    \item \textbf{1 Marketer}.
\end{itemize}

A equipa de IT, constituída por cinco desenvolvedores, foi organizada em duas subequipas: 
\begin{itemize}
    \item \textbf{3 estudantes} dedicados ao desenvolvimento de \textit{Backend};
    \item \textbf{2 estudantes} dedicados ao desenvolvimento de \textit{Frontend}.
\end{itemize}

Esta divisão teve como objetivo garantir um maior avanço na lógica de negócio durante a fase inicial do projeto. 
Numa etapa posterior, quando a lógica estivesse próxima da sua conclusão, alguns dos estudantes poderiam \textbf{transitar} para a subequipa de \textit{Frontend}, focando-se então na criação da \textbf{interface de utilizador}.



\subsection{Metodologia de trabalho}

Para uma melhor organização do trabalho, foi adotada a metodologia Scrum, com sprints de duas semanas de duração. Além disso, foi estabelecida, por consenso do grupo, a realização de reuniões semanais com o objetivo de atualizar o progresso das tarefas e promover um ambiente de trabalho mais colaborativo e comunicativo. Estas reuniões não possuiam uma data definida tendo em conta os fusos horarios de cada um dos membros

\subsubsection{Divisão de sprints}

A tabela \ref{tab:organizacao_sprints} demonstra de sprints que foi escolhida pela equipa durante a primeira semana de \textit{kickoff} do projeto.

%tab:organizacao_sprints
\begin{table}[h!tbp]
    \centering
    \caption{Organização de sprints definida pela equipa}
    \begin{tabular}{llll}
        Sprint & Semanas & Data de início & Data de fim \\\midrule
        1      & 1-2     & 24/02          & 02/03       \\\midrule
        2      & 3-4     & 10/03          & 23/03       \\\midrule
        3      & 5-6     & 24/03          & 06/04       \\\midrule
        4      & 7-8     & 07/04          & 20/04       \\\midrule
        5      & 9-10    & 21/04          & 04/05       \\\midrule
        6      & 11-12   & 05/05          & 18/05       \\\midrule
        7      & 13-14   & 19/05          & 01/06       \\\midrule
        8      & 15-16   & 02/06          & 15/06       \\\bottomrule
    \end{tabular}
    
    \label{tab:organizacao_sprints}
\end{table}
