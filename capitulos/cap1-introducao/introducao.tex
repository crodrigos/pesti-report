\chapter{Introdução}
\label{chap:introducao}

\section{Enquadramento}
\label{sec:introducao_enquadramento}

Este relatório foi desenvolvido com base num projeto enquadrado no âmbito da unidade curricular de Projeto Estágio (PESTI) da Licenciatura em Engenharia Informática (LEI) do Instituto Superior de Engenharia do Porto (ISEP)

Este projeto ocorreu no enquadramento do Projeto BlendED (também referido como Blended4Future ou BlendedMobility\footnote{https://blendedmobility.com}). Este curso dá a estudantes a oportunidade de desenvolverem as suas \textit{soft} e \textit{hard skills} num projeto com alunos de diferentes culturas e países trabalhando num projeto para empresas reais

\subsection{Apresentação da organização}

BlendedMobility é uma iniciativa educativa internacional que promove o desenvolvimento de projetos colaborativos no contexto do ensino híbrido. Esta visa combinar experiências de aprendizagem presencial e online, proporcionando aos estudantes uma formação mais flexível, personalizada e orientada para a prática.

O programa reúne alunos de diferentes universidades europeias que, ao longo de quatro meses, trabalham em conjunto no desenvolvimento de projetos reais para empresas parceiras. A metodologia adotada assenta em práticas ativas e colaborativas, potenciando competências essenciais como trabalho em equipa, comunicação intercultural e resolução de problemas num ambiente profissional simulado.

O percurso inicia-se com uma semana presencial no Instituto Universitário da Maia (ISMAI), onde as equipas se conhecem, recebem o enquadramento do projeto e planeiam as etapas de trabalho. O desenvolvimento dos projetos decorre num regime híbrido, combinando sessões online e trabalho autónomo. Ao final do ciclo, todas as equipas reúnem-se presencialmente na Universidade de Trier, na Alemanha, para apresentar os resultados dos seus projetos a um painel de avaliadores e representantes das empresas envolvidas.


\section{Descrição do Problema}
\label{sec:introducao_descproblema}

O Website do curso Blended4Future estava muito àquem do espectado, muitos elementos não seguiam um design inconsistente e antiquado, ou não estavam completamente funcionais. 

A organização desejava uma plataforma onde se pudesse autmaticamente adicionar projetos, alunos, universidades e empresas em uma só plataforma. Por tal foi colocada um proposta de desenvolvimento de uma nova aplicação web que substituiria esta anterior. 



\section{Objetivos}

A aplicação web a desenvolver deverá incluir um sistema de autenticação com diferenciação entre perfis de utilizador, nomeadamente administradores e utilizadores comuns, assegurando uma gestão adequada de permissões. 

Adicionalmente, deverá permitir a criação de novos projetos e a associação de diferentes intervenientes a cada um, consoante o seu papel. Para além disso, deverá ser implementada uma biblioteca de projetos, acessível a qualquer utilizador da plataforma, onde estarão disponíveis todos os projetos desenvolvidos no âmbito do curso, promovendo a sua consulta e divulgação.

A interface da aplicação deverá seguir as diretrizes de design definidas previamente por um membro da equipa dedicado ao design visual da aplicação.


\section{Abordagem}

A equipa era composta por um grupo de alunos de várias universidades europeias, com a seguinte constituição:

\begin{itemize}
\item 5 Desenvolvedores
\item 1 Designer
\item 1 Marketer
\end{itemize}

A equipa de IT de 5 desenvolvedores foi dividida em duas. Uma dedicada ao \textit{Frontend} e outra ao \textit{Backend}.

Para uma melhor organização do trabalho, foi adotada a metodologia Scrum, com sprints de duas semanas de duração. Além disso, foi estabelecida, por consenso do grupo, a realização de reuniões semanais com o objetivo de atualizar o progresso das tarefas e promover um ambiente de trabalho mais colaborativo e comunicativo.