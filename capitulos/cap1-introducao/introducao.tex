\chapter{Introdução}
\label{chap:introducao}

\section{Enquadramento}
\label{sec:introducao_enquadramento}

Este relatório foi desenvolvido com base num projeto enquadrado no âmbito da unidade curricular de Projeto Estágio (PESTI) da Licenciatura em Engenharia Informática (LEI) do Instituto Superior de Engenharia do Porto (ISEP)

Este projeto ocorreu no enquadramento do Projeto BlendED (também referido como Blended4Future ou BlendedMobility\footnote{https://blendedmobility.com}). Este curso dá a estudantes a oportunidade de desenvolverem as suas \textit{soft} e \textit{hard skills} num projeto com alunos de diferentes culturas e países trabalhando num projeto para empresas reais

\subsection{Apresentação da organização}

BlendedMobility é uma iniciativa educativa internacional que promove o desenvolvimento de projetos colaborativos no contexto do ensino híbrido. Esta visa combinar experiências de aprendizagem presencial e online, proporcionando aos estudantes uma formação mais flexível, personalizada e orientada para a prática.

O programa reúne alunos de diferentes universidades europeias que, ao longo de quatro meses, trabalham em conjunto no desenvolvimento de projetos reais para empresas parceiras. A metodologia adotada assenta em práticas ativas e colaborativas, potenciando competências essenciais como trabalho em equipa, comunicação intercultural e resolução de problemas num ambiente profissional simulado.

O percurso inicia-se com uma semana presencial no Instituto Universitário da Maia (ISMAI), onde as equipas se conhecem, recebem o enquadramento do projeto e planeiam as etapas de trabalho. O desenvolvimento dos projetos decorre num regime híbrido, combinando sessões online e trabalho autónomo. Ao final do ciclo, todas as equipas reúnem-se presencialmente na Universidade de Trier, na Alemanha, para apresentar os resultados dos seus projetos a um painel de avaliadores e representantes das empresas envolvidas.


\section{Descrição do Problema}
\label{sec:introducao_descproblema}

O Website do curso Blended4Future estava muito aquém do espectado, muitos elementos seguiam um design inconsistente e antiquado, ou não estavam completamente funcionais. 

A organização desejava uma plataforma onde se pudesse automaticamente adicionar projetos, alunos, universidades e empresas em uma só plataforma. Por tal foi colocada um proposta de desenvolvimento de uma nova aplicação web que substituiria esta anterior. 

Nesta aplicação, estudantes professores e representantes de empresas poderiam ver os projetos em que estavam envolvidos, fazer posts sobre os seus respetivos projetos.

Foi ainda requisitado uma maneira de ver todas a edições do Blended4Future e todos o projetos neste envolvido

\section{Objetivos}

A aplicação web a desenvolver deverá incluir um sistema de autenticação com diferenciação entre perfis de utilizador, nomeadamente administradores e utilizadores comuns, assegurando uma gestão adequada de permissões. 

Adicionalmente, deverá permitir a criação de novos projetos e a associação de diferentes intervenientes a cada um, consoante o seu papel. Para além disso, deverá ser implementada uma biblioteca de projetos, acessível a qualquer utilizador da plataforma, onde estarão disponíveis todos os projetos desenvolvidos no âmbito do curso, promovendo a sua consulta e divulgação.

A interface da aplicação deverá seguir as diretrizes de design definidas previamente por um membro da equipa dedicado ao design visual da aplicação.


\section{Abordagem}

\subsection{A equipa}

A equipa foi composta por um grupo de estudantes provenientes de várias universidades europeias, com a seguinte constituição:

\begin{itemize}
    \item 5 Desenvolvedores
    \item 1 Designer
    \item 1 Marketer
\end{itemize}

A equipa de IT, constituída por cinco desenvolvedores, foi organizada em duas subequipas: 
\begin{itemize}
    \item 3 estudantes dedicados ao desenvolvimento de \textit{Backend}
    \item 2 estudantes dedicados ao desenvolvimento de \textit{Frontend}, incluindo o autor deste relatório, que integrou esta subequipa para participar no desenvolvimento da interface de utilizador.
\end{itemize}

Esta divisão teve como objetivo garantir um maior avanço na lógica de negócio durante a fase inicial do projeto. 
Numa etapa posterior, quando a lógica estivesse próxima da sua conclusão, alguns dos estudantes poderiam transitar para a subequipa de \textit{Frontend}, focando-se então na criação da interface de utilizador.

\subsection{Metodologia de trabalho}

Para uma melhor organização do trabalho, foi adotada a metodologia Scrum, com sprints de duas semanas de duração. Além disso, foi estabelecida, por consenso do grupo, a realização de reuniões semanais com o objetivo de atualizar o progresso das tarefas e promover um ambiente de trabalho mais colaborativo e comunicativo. Estas reuniões não possuiam uma data definida tendo em conta os fusos horarios de cada um dos membros

\subsubsection{Divisão de sprints}

A tabela \ref{tab:organizacao_sprints} demonstra a organização de sprints que foi escolhida pela equipa durante a primeira semana de \textit{kickoff} do projeto.

%tab:organizacao_sprints
\begin{table}[h!tbp]
    \centering
    \caption{Organização de sprints definida pela equipa}
    \begin{tabular}{llll}
        Sprint & Semanas & Data de início & Data de fim \\\midrule
        1      & 1-2     & 24/02          & 02/03       \\\midrule
        2      & 3-4     & 10/03          & 23/03       \\\midrule
        3      & 5-6     & 24/03          & 06/04       \\\midrule
        4      & 7-8     & 07/04          & 20/04       \\\midrule
        5      & 9-10    & 21/04          & 04/05       \\\midrule
        6      & 11-12   & 05/05          & 18/05       \\\midrule
        7      & 13-14   & 19/05          & 01/06       \\\midrule
        8      & 15-16   & 02/06          & 15/06       \\\bottomrule
    \end{tabular}
    
    \label{tab:organizacao_sprints}
\end{table}

\subsection{Tecnologias Definidas}

Atendendo à heterogeneidade da equipa, composta por indivíduos com diferentes experiências e origens académicas, tornou-se imperativo definir consensualmente as tecnologias a utilizar no projeto. Para tal, procedeu-se ao levantamento sistemático das ferramentas previamente utilizadas pelos elementos da equipa, tendo-se optado, em média, pelas tecnologias mais frequentemente referidas e consideradas mais adequadas à criação de interfaces de utilizador e à implementação de APIs do tipo REST.


\begin{table}[h!tbp]
    \centering
    
    \caption{Levantamento das tecnologias que os elementos da equipa já utilizaram}

    \begin{tabular}{ll}
        \toprule
        Tecnologia & Nº de referências \\
        \midrule
        Java & 5 \\
        Python & 4 \\
        CSS & 4 \\
        MySQL & 4 \\
        HTML & 5 \\
        SQL Server & 3 \\
        C & 2 \\
        JavaScript & 5 \\
        TypeScript & 2 \\
        C++ & 2 \\
        Bash Scripting & 1 \\
        R & 1 \\
        ReactTS & 1 \\
        C\# & 1 \\
        .NET (ASP) & 1 \\
        Entity Framework & 1 \\
        H2 & 1 \\
        MongoDB & 1 \\
        PHP & 1 \\
        \bottomrule
    \end{tabular}


    \label{tab:tecnologias-mais-utilizadas}

\end{table}

Houve uma clara preferência quanto ao uso de linguagens com tipagem forte. Esta permite a melhor rastreamento de erros e a garante o formato dos objetos no programa. Esta preferência teve como base principal a inexperiência de vários elementos da equipa.

A tabela \ref{tab:tecnologias-mais-utilizadas} demonstra este levantamento que permitiu à equipa fundamentar a decisão sobre as tecnologias a adoptar:

\begin{enumerate}
    \item \textbf{Java}: Linguagem de programação orientada a objetos amplamente utilizada no desenvolvimento de aplicações empresariais e sistemas robustos. A sua integração com o ecossistema Spring possibilita a criação eficiente de APIs REST, suportando a escalabilidade e manutenção do software. A sua tipagem forte permite que erros sejam mais fáceis de evitar, principalmente para desenvolvedores com pouca experiência.  
        \begin{enumerate}
            \item \textbf{Spring}: Framework que simplifica o desenvolvimento de aplicações Java, especialmente no contexto de serviços web RESTful. Permite a gestão de pedidos HTTP, serialização de dados em JSON e integração com múltiplas bases de dados, favorecendo a modularidade e segurança das aplicações
        \end{enumerate}
    \item \textbf{TypeScript}: Linguagem de programação baseada em JavaScript, que integra tipagem estática e recursos avançados, facilitando o desenvolvimento de interfaces de utilizador complexas e a manutenção de grandes bases de código. A sua adoção é relevante na construção de aplicações web modernas e escaláveis.
    \begin{enumerate}
        \item \textbf{React TS}: Framework de desenvolvimento frontend baseada em componentes, que alia a flexibilidade do React à segurança oferecida pela tipagem forte do TypeScript. Esta combinação permite construir interfaces de utilizador dinâmicas, reutilizáveis e com maior robustez, suportando os modernos paradigmas de desenvolvimento web.
    \end{enumerate}
    \item \textbf{MySQL}: Sistema de gestão de bases de dados relacional de código aberto, amplamente utilizado em projetos de engenharia de software pela sua confiabilidade, eficiência no processamento de dados e compatibilidade com múltiplas linguagens de programação.
\end{enumerate}


% REF: tab:tecnologias-mais-utilizadas


