% ==================================================
%   Encoding & Fonts
% ==================================================
\usepackage[utf8]{inputenc}   % Input encoding
\usepackage[T1]{fontenc}      % Output encoding (hyphenation, accented chars)
\usepackage{lmodern}          % Latin Modern font
\usepackage{inconsolata}      % Monospaced font (good for code)

% ==================================================
%   Colors
% ==================================================
\usepackage{xcolor}           
\definecolor{pblue}{rgb}{0.13,0.13,1}
\definecolor{pgreen}{rgb}{0,0.5,0}
\definecolor{pred}{rgb}{0.9,0,0}
\definecolor{pgrey}{rgb}{0.46,0.45,0.48}

% ==================================================
%   Graphics & Figures
% ==================================================
\usepackage{graphicx}         % For including images
\usepackage{tikz}             % For programmatic graphics
% \usetikzlibrary{arrows}     % Uncomment if you need arrow styles
\usepackage{pgfplots}         % For plots
\pgfplotsset{compat=newest}
\usepackage{fvextra} % extensão de fancyvrb

% ==================================================
%   Tables
% ==================================================
\usepackage{booktabs}         % Professional table rules
\usepackage{tabularx}         % Tables with variable-width columns
\usepackage{makecell}         % Line breaks inside cells
\renewcommand\theadfont{\bfseries\normalsize}
\usepackage{multirow}         % Multirow cells
\usepackage{tabularray}       % Modern table package
\usepackage{longtable}        % Tables across pages
\usepackage{pdflscape}        % Landscape pages
\usepackage{geometry}         % Page geometry
\usepackage{calc}             % For calculations in table widths

% ==================================================
%   Lists
% ==================================================
\usepackage{enumitem}
\newcommand{\tablistcommand}{% eliminate vertical space before/after itemize
  \leavevmode\par\vspace{-\baselineskip}
}

% ==================================================
%   Code Listings
% ==================================================
\usepackage{listings}
\usepackage{fancyvrb}         % Makes listings more flexible
\usepackage{caption}          % Better captions

% -- General Java style
\lstset{
    language=Java,
    basicstyle=\ttfamily\footnotesize,
    showspaces=false,
    showtabs=false,
    breaklines=true,
    showstringspaces=false,
    breakatwhitespace=true,
    commentstyle=\color{pgreen},
    keywordstyle=\color{pblue},
    stringstyle=\color{pred},
    moredelim=[il][\textcolor{pgrey}]{$$},
    moredelim=[is][\textcolor{pgrey}]{\%\%}{\%\%},
    numbers=left,
    stepnumber=1,
    numbersep=10pt,
    tabsize=4,
    keepspaces=true,
    frame=tb,
    captionpos=b,
    aboveskip=1em,
    belowskip=1em,
    escapeinside=||
}

% -- Disable/enable line numbers in listings
\let\origthelstnumber\thelstnumber
\makeatletter
\newcommand*\Suppressnumber{%
  \lst@AddToHook{OnNewLine}{%
    \let\thelstnumber\relax%
    \advance\c@lstnumber-\@ne\relax%
  }%
}
\newcommand*\Reactivatenumber{%
  \lst@AddToHook{OnNewLine}{%
    \let\thelstnumber\origthelstnumber%
    \advance\c@lstnumber\@ne\relax
  }%
}
\makeatother

% -- Bash style
% =========================
% YAML literal
% =========================
\lstdefinestyle{yamlstyle}{
    basicstyle=\ttfamily\footnotesize,
    language=,
    showstringspaces=false,
    breaklines=true,
    frame=tb,
    numbers=left,
    numbersep=10pt,
    stepnumber=1,
    tabsize=2,
    keepspaces=true,
    captionpos=b,
    escapeinside={}{},        % desativa escape de LaTeX
    literate=*               % faz todos os caracteres literais
             {á}{{á}}1
             {é}{{é}}1
             {í}{{í}}1
             {ó}{{ó}}1
             {ú}{{ú}}1
             {Á}{{Á}}1
             {É}{{É}}1
             {Í}{{Í}}1
             {Ó}{{Ó}}1
             {Ú}{{Ú}}1
             {à}{{à}}1
             {À}{{À}}1
             {ã}{{ã}}1
             {Ã}{{Ã}}1
             {õ}{{õ}}1
             {Õ}{{Õ}}1
             {ç}{{ç}}1
             {Ç}{{Ç}}1
             {~}{{\textasciitilde}}1
             {^}{{\^}}1
             {_}{{\_}}1
             {:}{{:}}1
             {|}{{|}}1
             {*}{{*}}1
             {>}{{>}}1
             {-}{{-}}1
             {+}{{+}}1
             {<}{{<}}1
             {=}{{=}}1
             {&}{{&}}1
             {@}{{@}}1
             {\#}{{\#}}1
             {$}{{$}}1
             {\\}{{\textbackslash}}1
}

% =========================
% Bash literal
% =========================
\lstdefinestyle{bashstyle}{
    basicstyle=\ttfamily\footnotesize,
    language=,
    showstringspaces=false,
    breaklines=true,
    frame=tb,
    numbers=left,
    numbersep=10pt,
    stepnumber=1,
    tabsize=2,
    keepspaces=true,
    captionpos=b,
    escapeinside={}{},        % desativa escape de LaTeX
    literate=*               % faz todos os caracteres literais
             {á}{{á}}1
             {é}{{é}}1
             {í}{{í}}1
             {ó}{{ó}}1
             {ú}{{ú}}1
             {Á}{{Á}}1
             {É}{{É}}1
             {Í}{{Í}}1
             {Ó}{{Ó}}1
             {Ú}{{Ú}}1
             {à}{{à}}1
             {À}{{À}}1
             {ã}{{ã}}1
             {Ã}{{Ã}}1
             {õ}{{õ}}1
             {Õ}{{Õ}}1
             {ç}{{ç}}1
             {Ç}{{Ç}}1
             {~}{{\textasciitilde}}1
             {^}{{\^}}1
             {_}{{\_}}1
             {:}{{:}}1
             {|}{{|}}1
             {*}{{*}}1
             {>}{{>}}1
             {-}{{-}}1
             {+}{{+}}1
             {<}{{<}}1
             {=}{{=}}1
             {&}{{&}}1
             {@}{{@}}1
             {\#}{{\#}}1
             {$}{{$}}1
             {\\}{{\textbackslash}}1
             {(}{{(}}1
             {)}{{)}}1
}


% -- JavaScript style
\lstdefinestyle{Javascript}{
    language=JavaScript,
    basicstyle=\ttfamily\small,             
    keywordstyle=\color{blue}\bfseries,     
    commentstyle=\color{gray}\itshape,      
    stringstyle=\color{orange},             
    numberstyle=\tiny\color{gray},          
    numbers=left,                           
    stepnumber=1,
    numbersep=10pt,
    tabsize=2,
    breaklines=true,                        
    showstringspaces=false,
    frame=tb,                               
    captionpos=b,
    aboveskip=1em,
    belowskip=1em,
    morekeywords={let,const,var,function,return,if,else,for,while,do,switch,case,break,continue,new,try,catch,finally,throw,typeof,instanceof,this,class,extends,super,import,export,default,async,await}
}

% ==================================================
%   Bibliography & References
% ==================================================
\usepackage[backend=biber,style=numeric]{biblatex}
\addbibresource{bibliography.bib}



% ==================================================
%   Glossaries & Acronyms
% ==================================================
\usepackage[acronym]{glossaries}
\usepackage[automake=true]{glossaries-extra}

% ==================================================
%   Miscellaneous
% ==================================================
\usepackage{framed}           % Framed environments
\usepackage{bookmark}         % Better PDF bookmarks
