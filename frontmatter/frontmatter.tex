% we include the glossary here (frontmatter is included with \input, so this command is as if it was in main.tex)


\frontmatter % Use roman page numbering style (i, ii, iii, iv...) for the pre-content pages

%% PLACE THIS IN PREAMBLE PLS!!!!

%All acronyms must be written in this file. 
\makeglossaries

\newacronym{RTS}{RTS}{Real-Time System}

\newacronym{API}{API}{Application Programming Interface}
\newacronym{JPA}{JPA}{Java Persistence API}
\newacronym{ORM}{ORM}{Object/Relational Mapping}
\newacronym{IoC}{IoC}{Inversion of Control}
\newacronym{DI}{DI}{Dependency Injection}
\newacronym{HTTP}{HTTP}{Hypertext Transfer Protocol}
\newacronym{TCP}{TCP}{Transmission Control Protocol}

\newglossaryentry{NGINX}{
    name=\textit{NGINX},
    description={servidor \textit{web}}
}

\newglossaryentry{Build}{
    name=\textit{build},
    description={a transformação final de codigo de maior nivel para código legível pela máquina}
}

\newglossaryentry{Docker} {
    name=\textit{docker},
    description={uma plataforma para construir e disponbilizar aplicações a partir de containers}
}

\newglossaryentry{Image}{
    name=\textit{image},
    description={unidade de software que junta código, depêndencias e kernel de sistem em um só ficheiro. Em português, Imagem.}
}

\newglossaryentry{Spring}{
    name=\textit{Spring},
    description={framework Java}
}

\newglossaryentry{Container}{
    name=\textit{container},
    description={unidade de software \gls{image}, em execução}
}

\newglossaryentry{Hibernate}{
    name=\textit{Hibernate},
    description={\textit{Framework} Java responsável pelo mapeamento de um objeto de domínio para uma base de dados relacional }
}



\pagestyle{plain} % Default to the plain heading style until the thesis style is called for the body content
\setcounter{secnumdepth}{3}

%----------------------------------------------------------------------------------------
%	TITLE PAGE
%----------------------------------------------------------------------------------------

\maketitlepage


%----------------------------------------------------------------------------------------
%	ABSTRACT PAGE
%----------------------------------------------------------------------------------------

\begin{abstract}

% here you put the abstract in the main language of the work.

Trabalhos escritos em língua Inglesa devem incluir um resumo alargado com cerca de 1000 palavras, ou duas páginas.

Se o trabalho estiver escrito em Português, este resumo deveria ser em língua Inglesa, com cerca de 200 palavras, ou uma página.

Para alterar a língua basta ir às configurações do documento no ficheiro \file{main.tex} e alterar para a língua desejada ('english' ou 'portuguese')\footnote{Alterar a língua requer apagar alguns ficheiros temporários; O target \keyword{clean} do \keyword{Makefile} incluído pode ser utilizado para este propósito.}. Isto fará com que os cabeçalhos incluídos no template sejam traduzidos para a respetiva língua.

\end{abstract}

%----------------------------------------------------------------------------------------
%	ACKNOWLEDGEMENTS (optional)
%----------------------------------------------------------------------------------------

\begin{acknowledgements}

The optional Acknowledgment goes here\ldots Below is an example of a humorous acknowledgment.

"I'd also like to thank the Van Allen belts for protecting us from the harmful solar wind, and the earth for being just the right distance from the sun for being conducive to life, and for the ability for water atoms to clump so efficiently, for pretty much the same reason. Finally, I'd like to thank every single one of my forebears for surviving long enough in this hostile world to procreate. Without any one of you, this book would not have been possible." in "The Woman Who Died a Lot" by Jasper Fforde.

\end{acknowledgements}

%----------------------------------------------------------------------------------------
%	LIST OF CONTENTS/FIGURES/TABLES PAGES
%-----------------------------------------------------------------------------------

\tableofcontents % Prints the main table of contents

\listoffigures % Prints the list of figures

\listoftables % Prints the list of tables

%\renewcommand{\listalgorithmname}{Lista de Algor\'itmos}
%\listofalgorithms % Prints the list of algorithms
%\addchaptertocentry{\listalgorithmname}

\renewcommand{\lstlistlistingname}{Lista de C\'odigo}
\lstlistoflistings % Prints the list of listings (programming language source code)
\addchaptertocentry{\lstlistlistingname}

%----------------------------------------------------------------------------------------
%	ACRONYMS / GLOSSARY
%----------------------------------------------------------------------------------------

%\renewcommand{\listacronymname}{Lista de Acr\'onimos}

\glsaddall

\printacronyms
\printglossary

%----------------------------------------------------------------------------------------
%	DONE
%----------------------------------------------------------------------------------------

\mainmatter % Begin numeric (1,2,3...) page numbering
\pagestyle{thesis} % Return the page headers back to the "thesis" style