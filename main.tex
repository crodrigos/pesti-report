
\documentclass[
11pt, % The default document font size, options: 10pt, 11pt, 12pt
%oneside, % Two side (alternating margins) for binding by default, uncomment to switch to one side (for drafting/reading purposes)
%english, % english for English;
portuguese,% for Portuguese; delete temporary files if you change language (e.g. 'make clean; make')
singlespacing, % Single line spacing, alternatives: onehalfspacing or doublespacing (for drafting/reading purposes)
%draft, % Uncomment to enable draft mode (no pictures, no links, overfull hboxes indicated)
%nolistspacing, % If the document is onehalfspacing or doublespacing, uncomment this to set spacing in lists to single
liststotoc, % Uncomment to add the list of figures/tables/etc to the table of contents (recommended)
%toctotoc, % Uncomment to add the main table of contents to the table of contents (not recommended)
parskip, % Add space between paragraphs (recommended)
%nohyperref, % Uncomment to not load the hyperref package (not recommended)
nohyperreflinkcolor, % hyperref links are not colored (comment to color links, for example to produce an electronic-only version)
headsepline, % Uncomment to get a line under the header
]{tmdei-style} % The class file specifying the document structure

\usepackage{tikz} % Required for creating graphics programmatically (can be removed if not used)
%\usetikzlibrary{arrows} % Required for fancy arrows in TiKZ graphics (can be removed if not used)

\usepackage{pgfplots} % Required for drawing high--quality function plots (can be removed if not used)
\pgfplotsset{compat=newest}

\usepackage[style=authoryear-comp,backend=biber]{biblatex} 

\addbibresource{mainbibliography.bib} % The filename of the bibliography

\usepackage[acronym]{glossaries} % Load glossaries package
\makeglossaries % build the glossary

%----------------------------------------------------------------------------------------
%	THESIS INFORMATION
%----------------------------------------------------------------------------------------

\thesistitle{Desenvolvimento de uma plataforma de gestão de projetos} % Your thesis title, this is used in the title, print it elsewhere with \ttitle

%\thesissubtitle{{[}Thesis Subtitle{]}} % Your thesis title, this is used in the title, print it elsewhere with \tsubtitle

\author{Carlos Santos} % Your name, this is used in the title page, print it elsewhere with \authorname

\subjectarea{Cybersecurity And Systems Administration} % Specialisation area (Cybersecurity and Systems Administration; Data Engineering; Software Engineering;Games, Graphics and Interactive Systems; Information and Knowledge Systems), used in the title page, print it elsewhere with \areaname

\advisor{Professor Nuno Escudeiro} % Your advisors's (academic mentor) name, this is used in the title page, print it elsewhere with \advname

\supervisor{Dr. James \textsc{Smith}} % Your supervisor's (company, hosting institution mentor) name, this is used in the title page, print it elsewhere with \supname (comment, if no supervisor, or same as advisor)

%\coadvisor{Dr. Jack \textsc{Smith}} % Your co-advisors's name, this is used in the title page, print it elsewhere with \coadvname (comment, if no co-advisor)

%\cosupervisor{Dr. Jack \textsc{Smith}} % Your co-advisors's name, this is used in the title page, print it elsewhere with \cosupname (comment, if no co-supervisor)

% if committeepresident is defined, will add the thesis committee to the front page
%\committeepresident{Dr. Jonny Smith, Professor, DEI/ISEP} % Name of the president of the evaluation committee, print it elsewhere with \presidentname

%\committeemembers{Dr. Jaimie Smith, Professor, DEI/ISEP\\Dr. Jones Smith, Professor, DEI/ISEP\\Dr. Jagger Smith, Professor, DEI/ISEP} % Name of the evaluation committee members (up to four), print it elsewhere with \committee

\keywords{Keyword1, ..., Keyword6} % Please define up to 6 keywords that better describe your work, print it elsewhere with \keywordnames

\university{\href{http://www.university.com}{University Name}} % Your university's name and URL, this is used in the title page and abstract, print it elsewhere with \univname

\department{\href{http://dei.isep.ipp.pt}{Departamento de Engenharia Informatica}} % Your department's name and URL, this is used in the title page and abstract, print it elsewhere with \deptname

\thesisdate{Porto, Setembro, 2025} % thesis date,  print it elsewhere with \tdate

\hypersetup{pdftitle=\ttitle} % Set the PDF's title to your title
\hypersetup{pdfauthor=\authorname} % Set the PDF's author to your name
\hypersetup{pdfkeywords=\keywordnames} % Set the PDF's keywords to your keywords

\begin{document}

%----------------------------------------------------------------------------------------
%	FRONT MATTER
%----------------------------------------------------------------------------------------

% Include the frontmatter of your thesis here
% we include the glossary here (frontmatter is included with \input, so this command is as if it was in main.tex)


\frontmatter % Use roman page numbering style (i, ii, iii, iv...) for the pre-content pages

%% PLACE THIS IN PREAMBLE PLS!!!!

%All acronyms must be written in this file. 
\makeglossaries

\newacronym{RTS}{RTS}{Real-Time System}

\newacronym{API}{API}{Application Programming Interface}
\newacronym{JPA}{JPA}{Java Persistence API}
\newacronym{ORM}{ORM}{Object/Relational Mapping}
\newacronym{IoC}{IoC}{Inversion of Control}
\newacronym{DI}{DI}{Dependency Injection}
\newacronym{HTTP}{HTTP}{Hypertext Transfer Protocol}
\newacronym{TCP}{TCP}{Transmission Control Protocol}

\newglossaryentry{NGINX}{
    name=\textit{NGINX},
    description={servidor \textit{web}}
}

\newglossaryentry{Build}{
    name=\textit{build},
    description={a transformação final de codigo de maior nivel para código legível pela máquina}
}

\newglossaryentry{Docker} {
    name=\textit{docker},
    description={uma plataforma para construir e disponbilizar aplicações a partir de containers}
}

\newglossaryentry{Image}{
    name=\textit{image},
    description={unidade de software que junta código, depêndencias e kernel de sistem em um só ficheiro. Em português, Imagem.}
}

\newglossaryentry{Spring}{
    name=\textit{Spring},
    description={framework Java}
}

\newglossaryentry{Container}{
    name=\textit{container},
    description={unidade de software \gls{image}, em execução}
}

\newglossaryentry{Hibernate}{
    name=\textit{Hibernate},
    description={\textit{Framework} Java responsável pelo mapeamento de um objeto de domínio para uma base de dados relacional }
}



\pagestyle{plain} % Default to the plain heading style until the thesis style is called for the body content
\setcounter{secnumdepth}{3}

%----------------------------------------------------------------------------------------
%	TITLE PAGE
%----------------------------------------------------------------------------------------

\maketitlepage


%----------------------------------------------------------------------------------------
%	ABSTRACT PAGE
%----------------------------------------------------------------------------------------

%\begin{abstract}

% here you put the abstract in the main language of the work.

%Trabalhos escritos em língua Inglesa devem incluir um resumo alargado com cerca de 1000 palavras, ou duas páginas.

%Se o trabalho estiver escrito em Português, este resumo deveria ser em língua Inglesa, com cerca de 200 palavras, ou uma página.

%Para alterar a língua basta ir às configurações do documento no ficheiro \file{main.tex} e alterar para a língua desejada ('english' ou 'portuguese')\footnote{Alterar a língua requer apagar alguns ficheiros temporários; O target \keyword{clean} do \keyword{Makefile} incluído pode ser utilizado para este propósito.}. Isto fará com que os cabeçalhos incluídos no template sejam traduzidos para a respetiva língua.

%\end{abstract}

%----------------------------------------------------------------------------------------
%	ACKNOWLEDGEMENTS (optional)
%----------------------------------------------------------------------------------------

\begin{acknowledgements}

Ao professor Nuno Escudeiro, na qualidade de orientador, pelo convite para integrar este projeto e pela confiança demonstrada ao longo do mesmo.

Ao professor Ricardo Almeida, pelo apoio contínuo e valiosas orientações disponibilizadas durante o desenvolvimento do projeto.

À minha família e à minha namorada, pelo apoio incondicional, pelo carinho e atenção dedicados, e por me ouvirem e motivarem nos momentos mais difíceis.

\end{acknowledgements}

%----------------------------------------------------------------------------------------
%	LIST OF CONTENTS/FIGURES/TABLES PAGES
%-----------------------------------------------------------------------------------

\tableofcontents % Prints the main table of contents

\listoffigures % Prints the list of figures

\listoftables % Prints the list of tables

%\renewcommand{\listalgorithmname}{Lista de Algor\'itmos}
%\listofalgorithms % Prints the list of algorithms
%\addchaptertocentry{\listalgorithmname}

\renewcommand{\lstlistlistingname}{Lista de C\'odigo}
\lstlistoflistings % Prints the list of listings (programming language source code)
\addchaptertocentry{\lstlistlistingname}

%----------------------------------------------------------------------------------------
%	ACRONYMS / GLOSSARY
%----------------------------------------------------------------------------------------

%\renewcommand{\listacronymname}{Lista de Acr\'onimos}

% \glsaddall

% \printacronyms
% \printglossary

%----------------------------------------------------------------------------------------
%	DONE
%----------------------------------------------------------------------------------------

\mainmatter % Begin numeric (1,2,3...) page numbering
\pagestyle{thesis} % Return the page headers back to the "thesis" style

%----------------------------------------------------------------------------------------
%	MAIN BODY
%----------------------------------------------------------------------------------------

% Include the chapters of the thesis as separate folder for each chapter
% Uncomment the lines as you write the chapters

\chapter{Introdução}
\label{chap:introducao}

\section{Enquadramento}
\label{sec:introducao_enquadramento}

O projeto foi realizado em conjunto com o curso Blended Mobility.

O foco principal será o desenvolvimento de um projeto no âmbito do curso BlendED, que se caracteriza pela integração da aprendizagem em mobilidade num contexto de trabalho híbrido. Este curso promove a colaboração entre alunos de várias universidades internacionais permitindo a combinação de atividades presenciais e online, potencializando a flexibilidade e a personalização do processo de aprendizagem. Todas as equipas envolvidas terão de desenvolver, ao longo de quatro meses, um projeto para uma empresa parceira, aplicando metodologias ativas e colaborativas típicas do ensino híbrido.

Numa primeiro fase, todos os membros deslocaram-se se até ao Instituto Universitário da Maia (ISMAI), onde se deu uma semana para conhecer toda a equipa, contextualização do projeto e estruturação do trabalho para os seguintes meses. Durante os seguintes quatro meses ocorrereu o desenvolvimento do projeto. No final, todas a equipas irão reuiniram na Universidade de Trier, na Alemanha, para a apresentação final do que foi desenvolvido.



\section{Descrição do Problema}
\label{sec:introducao_descproblema}

O Website do curso Blended4Future estava muito àquem do espectado, muitos elementos não seguiam um design inconsistente e antiquado, ou não estavam completamente funcionais. 

A organização desejava uma plataforma onde se pudesse autmaticamente adicionar projetos, alunos, universidades e empresas em uma só plataforma. Por tal foi colocada um proposta de desenvolvimento de uma nova aplicação web que substituiria esta anterior. 



\section{Objetivos}

A aplicação web a desenvolver deverá incluir um sistema de autenticação com diferenciação entre perfis de utilizador, nomeadamente administradores e utilizadores comuns, assegurando uma gestão adequada de permissões. 

Adicionalmente, deverá permitir a criação de novos projetos e a associação de diferentes intervenientes a cada um, consoante o seu papel. Para além disso, deverá ser implementada uma biblioteca de projetos, acessível a qualquer utilizador da plataforma, onde estarão disponíveis todos os projetos desenvolvidos no âmbito do curso, promovendo a sua consulta e divulgação.

A interface da aplicação deverá seguir as diretrizes de design definidas previamente por um membro da equipa dedicado ao design visual da aplicação.



\section{Abordagem}

A equipa era composta por um grupo de alunos de várias universidades europeias, com a seguinte constituição:

\begin{itemize}
\item 5 Desenvolvedores
\item 1 Designer
\item 1 Marketer
\end{itemize}

Para uma melhor organização do trabalho, foi adotada a metodologia Scrum, com sprints de duas semanas de duração. Além disso, foi estabelecida, por consenso do grupo, a realização de reuniões semanais com o objetivo de atualizar o progresso das tarefas e promover um ambiente de trabalho mais colaborativo e comunicativo.

%----------------------------------------------------------------------------------------
%	BIBLIOGRAPHY
%----------------------------------------------------------------------------------------

\printbibliography[heading=bibintoc]

%----------------------------------------------------------------------------------------
%	THESIS CONTENT - APPENDICES
%----------------------------------------------------------------------------------------
% Include the appendices of the thesis as separate files from the Appendices folder
% Uncomment the lines as you write the Appendices
\begin{appendices}

% Appendix A

\chapter{Appendix Title Here} % Main appendix title

\label{AppendixA} % For referencing this appendix elsewhere, use \ref{AppendixA}

Write your Appendix content here.
%\input{appendices/appendixB}
%\input{appendices/appendixC}

\end{appendices}
%----------------------------------------------------------------------------------------

\end{document}
